\documentclass[sigchi, review]{acmart}

\usepackage{booktabs} % For formal tables

% Copyright
\setcopyright{none}
%\setcopyright{acmcopyright}
\setcopyright{acmlicensed}
%\setcopyright{rightsretained}
%\setcopyright{usgov}
%\setcopyright{usgovmixed}
%\setcopyright{cagov}
% \setcopyright{licensedcagov}
%\setcopyright{cagovmixed}
%\setcopyright{licensedothergov}

% DOI
\acmDOI{10.475/123_4}

% ISBN
\acmISBN{123-4567-24-567/08/06}

%Conference
\acmConference[CHI'19]{ACM SigCHI conference}{April 2019}{ElPaso, Texas USA}
\acmYear{2019}
\copyrightyear{2019}

\acmPrice{15.00}
\usepackage{CJKutf8}

\usepackage{multicol}
\usepackage{multirow}
\usepackage{array}
\usepackage{booktabs}
\usepackage[utf8]{inputenc}

\begin{document}
\begin{CJK*}{UTF8}{gbsn}

\title{TeleCP: a software framework for 3D telepresence that adequately supports co-presence}

%\titlenote{Produces the permission block, and copyright information}
%\subtitle{Extended Abstract}
%\subtitlenote{The full version of the author's guide is available as \texttt{acmart.pdf} document}

\author{Leave Authors Anonymous}
\affiliation{
  \institution{Institute}
  \city{City}
  \state{Country}
}
\email{example@email.com}

\author{Leave Authors Anonymous}
\affiliation{
  \institution{Institute}
  \city{City}
  \state{Country}
}
\email{example@email.com}

\author{Leave Authors Anonymous}
\affiliation{
  \institution{Institute}
  \city{City}
  \state{Country}
}
\email{example@email.com}

% The default list of authors is too long for headers.
\renewcommand{\shortauthors}{B. Trovato et al.}


\begin{teaserfigure}
\centering
\includegraphics[width=\textwidth]{figures/figure_teaser.jpg}
\caption{The figure shows two applications supported by our framework. (a) is the Rock-Paper-Scissors game that requires a high temporal synchronicity. (b) is a remote chess game that shows a high level of spatial synchronicity.}
\label{fig:teaser}
\end{teaserfigure}

\begin{abstract}

We present TeleCP: a software framework for 3D telepresence that adequately supports co-presence. Based on some commercial hardware (Realsense, PCs and HTC Vive), the framework allows distributed users to communicate and interact with each other in the same virtual space at the same time. Co-presence consists of temporal synchronicity and spatial synchronicity. For temporal synchronicity, TeleCP achieves a low end-to-end delay of 50 ms with direct Ethernet connection. TeleCP also provides sync assistance to help the users perform simultaneously regardless of the unavoidable network delay. For spatial synchronicity, we render live 3D reconstruction in Head-Mounted Display (HMD). This approach provides the sense that multiple users share a 3D space. TeleCP further allows users to share props by merging similar physical objects in the two ends. In this paper, we introduce the implementation in details. The open-source project contains an Unity3D plugin that simplifies upper-layer application development. We describe an experiment with two tasks to validate the temporal and spatial synchronicity of TeleCP.

%3D Tele-Immersion (3DTI, e.g., Holoportation \cite{orts2016holoportation}) is the next generation of telecommunication that could become practical in the near future. However, the interaction possibilities of 3DTI have not been well discussed. The cost of developing and deploying a 3DTI system is extremely high. HCI researchers without expertise cannot easily acquire such a system. To fill the gap, we present an open-source, lightweight 3DTI system that supports a high level of co-presence. The system has two advantages. First, it is inexpensive (\$ 7000), easy to deploy and use. Second, according to the brainstorming of possible 3DTI applications, our support of co-presence is the key for new interaction possibilities in 3DTI. The co-presence we supported has four contents: rich visual information, spatial co-presence, high synchronicity, and shared props. In this paper, we introduce the system in details and describe a user study to validate the system.

\end{abstract}

%
% The code below should be generated by the tool at
% http://dl.acm.org/ccs.cfm
% Please copy and paste the code instead of the example below.
%
\begin{CCSXML}
<ccs2012>
<concept>
<concept_id>10003120.10003121.10003129.10011757</concept_id>
<concept_desc>Human-centered computing~User interface toolkits</concept_desc>
<concept_significance>500</concept_significance>
</concept>
</ccs2012>
\end{CCSXML}

\ccsdesc[500]{Human-centered computing~User interface toolkits}

\keywords{3D telepresence; co-presence; GPU}

\maketitle

\section{Introduction}

% [writing] First, provide some context to orient those readers who are less familiar with your topic and to establish the importance of your work.

% [paragraph] development of DIME in audio & 2D
The past centuries have witnessed the growth of communication technology. The invention of the telephone has saved a great deal of time and money by displacing meeting. Recently, Distributed Interactive Multimedia Environments (DIMEs) is getting popular. It provides convenience for teleconference \cite{marlow2016beyond}, tele-collaboration \cite{donovan2014understanding, avellino2015accuracy}, robotic telepresence \cite{jouppi2001robotic, misawa2015chameleonmask, neustaedter2016beam}, and so on. Development of communication technology is never separated from the studies of user experience. For example, delay of 150 ms provides a good user experience for most audio-mediated applications \cite{recommendation2003114, donovan2014understanding}. It has become an industrial standard that contributes to telephone network engineering \cite{itu2003recommendation}. 2D DIME also benefits from the studies of user experience, e.g., in the regions of pointing \cite{higuchi2015immerseboard, avellino2015accuracy}, distance perception \cite{boustila2015evaluation, alexandrova2010egocentric} and delay perception \cite{geerts2011we, tam2012video, schmitt2014influence}.

% [paragraph] rapid growth of 3DTI recently
3DTI emerged in the last past decades \cite{kurillo2008immersive, petit2010multicamera, maimone2011encumbrance, maimone2012real}. Microsoft Research's Holoportation \cite{orts2016holoportation} was impressive. They presented an end-to-end 3DTI pipeline with high-quality, real-time reconstructions of an entire space. Furthermore, parallel computing devices such as GPUs are getting more powerful. Immersive displays such as Head-Mounted Displays (HMDs) are becoming popular. In a word, both the improvements of algorithm and hardware make 3DTI hopeful to be practical in the near future.

% [writing] Second, state the need for your work, as an opposition between what the scientific community currently has and what it wants.

% [paragraph] what the scientific community currently has and what it wants
The focus of previous 3DTI works has been mainly on technical implementations. However, few works were conducted to study user experience in 3DTI. In particular, no work has been done to study users' perception of network delay in an advanced 3DTI system.

% [paragraph] the need of our work
Network delay is a crucial factor that affects user experience \cite{brunnstrom2013qualinet, schmitt2014asymmetric, schmitt2013qoe, wu2009quality}. The studies of delay perception is useful. Numerous works have been carried out to explore delay perception in telephone and 2D DIME. Given a specific task, noticeability and acceptance of delays are important factors to measure \cite{wu2009quality, schmitt2014influence, geerts2011we, schmitt2014asymmetric}. On the one hand, a network service should achieve the acceptable delay as far as possible. On the other hand, there is no need to improve the network while the delay is already unnoticeable. Beyond that, many other strategies were proposed to save network resources and improve the user experience.

% [paragraph] The difference between 3D and 2D - why should we rebuild the framework
Despite of the sufficient research in 2D, it is still necessary to rebuild the framework of delay perception in 3D. The reason is the large difference between 3DTI and 2D DIME: First, 3DTI can support more applications and improve some existing tasks in a more nature manner. We have to discuss them case by case; Second, 3DTI refers to a higher level of immersion, which offers more visual cues. Previous work \cite{tam2012video} have pointed out that video increase users' tolerance to delay. This effect may be enhanced in 3D.

% [writing] Third, indicate what you have done in an effort to address the need (this is the task).

% [paragraph] in this paper, we propse the framework
In this paper, we propose a conceptual framework of network delay perception in 3DTI. It levels delay requirements of 3DTI tasks into three classes: \emph{synchronous} tasks, \emph{audiovisual} tasks and \emph{visual only} tasks, which require network delays of about 50 ms, 200 ms and 400 ms respectively. We designed the framework through a comprehensive review on delay perception and 3DTI systems. For each level in our framework, we summarized suggestions on network engineering. To validate the framework, we first followed the mainstream works to build up our 3DTI systems. Then, we conducted a controlled study to illustrate our framework. To our knowledge, we are the first to investigate users' perception of network delay in a full 3D tele-immersion system.

% [paragraph] Contributions
The contribution of our work is threefold: First, the framework infers a significant change of network delay perception in 3D. We recommend that the 3DTI developers should first assess his application through our framework; Second, we summarize suggestions on network engineering to cope with different level of tasks. These suggestions can help saving network resource and improving the user experience; Third, our project is open-source [?]. We give necessary explanation in the system overview to make sure that the readers can easily build up a similar system.

% [writing] Finally, preview the remainder of the paper to mentally prepare readers for its structure, in the object of the document.

% [paragraph] paper structure
We construct the paper as follow: In section 2, we present our framework. In section 3, we give an overview of our experimental system. In section 4, we describe the controlled study to illustrate our framework. In section 5, we supplement related works on system implementation and existing works on 3DTI delay perception. In section 6, we discuss the limitation of our work. At the end we draw the conclusion.


\section{Related Work}

% [paragraph] related work structure
In this section, we first review 3D tele-immersion techniques. We summarize the necessary components of current 3DTI systems to guide our implementation. Then, we have a look about existing studies on delay in 3DTI. Much more related researches were conducted in audiovisual DIME (Distributed Interactive Multimedia Environments). We discuss them later to help forming our framework.

\subsection{3D Tele-immersion}

% [paragraph] introduction of 3D Tele-immersion
Optimal techniques toward 3DTI became clear in the last decade. Basically, a 3DTI system requires three processes: reconstruction, transmission and rendering \cite{fuchs2014immersive}. For 3D reconstruction, the volumetric algorithm has become mainstream. We applied TSDF Volume \cite{curless1996volumetric} and Marching Cubes \cite{lorensen1987marching} to fuse depth images into a polygonal mesh. We do not focus on transmission as \cite{beck2013immersive, pece2011adapting} did, but use 10 Gigabit optical fiber between computers instead. For rendering, we applied the head-mounted display (HTC Vive) because of its on-going growth. In this subsection, we review previous works of 3D reconstruction and rendering in details.

\subsubsection{3D Reconstrucion}

% [paragraph] cameras sea (<2000)
In early works, researchers used an array of cameras to capture dynamic scenes \cite{kanade1997virtualized, fuchs1994virtual}. For a given camera view, these systems create a polygonal model that will look correct. They do not actually construct a 3D model.

% [paragraph] toward volumetric reconstruction 2000 ~ 2010
TELEPORT \cite{gibbs1999teleport} composites video-textured surfaces within 3D geometric models. It uses only one camera. In 2002, researchers started to design virtual 3D environment with multiple cameras \cite{gross2003blue, towles20023d}. However, their 3D reconstruction result was only point cloud. In 2008, Kurillo et al. presented a framework for remote collaboration and training of physical activities \cite{kurillo2008immersive}. This work tried a reconstruction method with triangulation, but only reached the frame rate of about 5-7 FPS. \cite{loop2013real} and \cite{petit2010multicamera} for the first time presented compelling real-time reconstruction techniques with multiple cameras. However, the lack of depth dimension indicated their modeling with only silhouette boundaries.

% [paragraph] toward volumetric reconstruction 2010 ~ 2015
Researchers achieved the real-time performance of high-quality reconstruction in the last decade. In October 2011, Maimone et al. presented a 3DTI system with Kinects \cite{maimone2011encumbrance}. They developed a pixel-based mesh generation algorithm and reached a frame rate of 30 FPS. This work was followed by Beck et al.'s group-to-group telepresence system \cite{beck2013immersive}. In the same month, however, Microsoft introduced KinectFusion \cite{izadi2011kinectfusion} based on volumetric method. They described a novel GPU-based pipeline and achieved a better reconstruction quality. In the next year (2012), Maimone et al. also turned to volumetric methods \cite{maimone2012real} to improve the quality. A huge amount of works improved 3D reconstruction within the same framework as KinectFusion, in the region of scale \cite{niessner2013real, chen2013scalable}, noise reduction \cite{khoshelham2012accuracy, nguyen2012modeling, newcombe2015dynamicfusion} and so on.

% [paragraph] toward fusion4D and its drawback (2016)
In 2016, Microsoft proposed a new pipeline named Fusion4D \cite{dou2016fusion4d}, which is highly robust to occlusions, large frame-to-frame motions, and topology changes. "The fourth dimension" is the time dimension, indicating that it leverages temporally coherence of physical scenes. In the same year, Microsoft integrated fusion4D into their 3DTI system Holoportation \cite{orts2016holoportation}. However, Fusion4D is extremely complex and not open-source. Even with costly devices, Holoportation has an end-to-end latency of 80ms, which can not be ignored in our study. In this paper, we apply a 3D reconstruction method similar to \cite{maimone2012real} (2012) for responsiveness.

\subsubsection{3D Rendering}

% [paragraph] categories of 3D rendering, end light field displays here
Rendering techniques in 3DTI systems can be mainly divided into three categories: light field displays, spatially immersive displays (SIDs) and head-mounted displays (HMDs). The light field displays \cite{jones2007rendering, jurik2011prototyping, kim2012telehuman, gotsch2018telehuman2} suffers from low resolution because neither computing nor rendering devices can support high-quality 4d light fields. SIDs were earlier, while HMDs are becoming popular nowadays.

% [paragraph] spatially immersive displays - earlier
Around year 2000, SIDs had become increasing significant \cite{gross2003blue}. CAVE \cite{cruz1993surround} is a typical SIDs system, which consists of surround-screen projection. Users wear 3D glasses in a CAVE. Most 3DTI systems at that time applied rendering techniques similar to CAVE \cite{gibbs1999teleport, towles20023d, gross2003blue, kurillo2008immersive, benko2012miragetable}. CAVE was design to support the one-to-many presentation. Latter researchers improved it for multi-user by using polarization or time-sharing \cite{frohlich2005implementing, kulik2011c1x6, guan2018two}. Multi-user SID was used by an immersive group-to-group telepresence \cite{beck2013immersive}. There is also a simplified technique called head-tracked auto-stereo display \cite{benko2014dyadic, jones2014roomalive}, which allows 3D view without glasses. Some 3DTI system \cite{maimone2011encumbrance, maimone2012real, pejsa2016room2room} used it for rendering. However, these glasses-free systems have to abandon the benefit of stereoscopy.

% [paragraph] head-mounted displays - popular now
Recently, HMDs are becoming popular. More 3DTI systems tend to apply HMDs for 3D rendering \cite{orts2016holoportation, maimone2013general, lindlbauer2018remixed, smith2018communication}. HMDs are basically cheaper and easier to deploy compared to SIDs. Another superiority of HMDs is their ability to support co-located collaboration \cite{maimone2013general, orts2016holoportation}, i.e., users feel like exactly in the same place. For comparison, SIDs do not support rendering in full 3D, with which a 'window' separate users into two virtual spaces. In 2018, Microsoft proposed Remixed Reality \cite{lindlbauer2018remixed}. This approach combines the benefits of augmented reality and virtual reality using 3D reconstruction and VR HMD. Users can not only see their environment but can also apply changes to it. Finally, we applied the head-mounted VR (HTC Vive) for rendering.

\subsection{Delay Perception in 3DTI}

% [paragraph] introduction of delay perception in telepresence
User experience often relates to QoS (quality of service) including delay, bandwidth, jitter and packet loss \cite{donovan2014understanding}. Previous works have found that delay is one of the most crucial factors determining user experience in telepresence \cite{vogel1995distributed, brunnstrom2013qualinet, schmitt2014asymmetric, schmitt2013qoe}. For telephone, 150ms has been established as an industry standard for an acceptable delay \cite{rec2003g, percy1999understanding}. Also, a huge amount of related works have been conducted in audiovisual DIME. 

% [paragraph] what is different in 3D?
3DTI is quite different from audiovisual DIME: first, high level of immersion offers more cues, i.e., users may be more sensitive to delay in 3DTI; second, with abundant sensory stimuli, users are more tolerant to delay \cite{tam2012video}; third, 3DTI can support much more possible applications that we have to discuss them case by case. Thus, we have to rebuild the theoretical framework of delay perception in 3DTI.

% [paragraph] existing work of delay perception in 3DTI
In the area of 3DTI, most works focus on algorithm and pipeline. Negative impacts of large delay are widely reported \cite{beck2013immersive, gibbs1999teleport, maimone2011encumbrance, kurillo2008immersive, raghuraman2015distortion}. However, only a few works were conducted to study delay perception in 3DTI \cite{wu2009quality, wu2010m, huang2012towards}. These works do not exactly focus on delay perception. Moreover, they are limited by single scenario and immature techniques, e.g., the 2D screen was used to display 3D scenes. In this paper, we explore delay perception in a full 3D tele-immersion. We consider various scenarios and finally form a framework to understand this problem.


\section{A Framework of Delay Perception in 3DTI}

%We present a conceptual framework of delay perception in 3DTI. \textbf{In this framework, 3DTI applications are divided into three levels according to their delay requirement. 这句话不清楚,很想知道什么是一个level,但是没有满足预期,猜也猜不到}. Suggestions are given to network engineering for each level. \textbf{Furthermore, we have designed an experiment to illustrate the framework.} Readers can reconstruct the experiment to precisely measure the delay requirement of a specific application.

% The structure of our framework
We present a conceptual framework of delay perception in 3DTI. The framework divides 3DTI tasks into three levels: tasks with \emph{synchronous interaction}, \emph{conversation} and \emph{only visual feedback}. Correspondingly, the three levels of tasks call for high, middle and low requirement of network delay. For each level, the framework also gives suggestions to improve the perceived quality of network.

The framework is supported a comprehensive reviews of previous works. Delay perception is a well-understood research question in audiovisual DIME. \textbf{Our framework partly relies on previous theories and study results(这句话不清楚)}. However, there is a large difference of delay perception between 3DTI and audiovisual DIME. Thus, we take the features of 3DTI into account in order to form our framework.

Noticeability and tolerance of delay are two important factors that were measured by most studies \cite{wu2009quality, schmitt2014influence, geerts2011we, schmitt2014asymmetric}. Some studies also focus on users' perception of overall network quality. These metrics are users' subjective rating in a specific task, and with a specific delay. In our study, we assess subjective feedback via questionnaires similar to \cite{schmitt2014influence}. The questionnaire is on a 5-point Likert scale (Table \ref{tab:table_questionnaire}).

\begin{table} [!htbp]
\begin{tabular}{|p{0.25\columnwidth}|p{0.35\columnwidth}|p{0.3\columnwidth}|}
\hline 
Label & Question & Scale \\
\hline
quality & How do you feel during the experiment? &Bad <--> Excellent \\
\hline
noticeability & Can you perceive the delay in the connection? & Very much <--> Not at all \\
\hline
tolerance & To what extent where you annoyed by the delay? & Severe annoyance <--> No annoyance \\
\hline
\end{tabular}
\caption{Questions and scale.}
\label{tab:table_questionnaire}
\end{table}

A 3DTI developer may expect a certain recommended delay for his application. However, previous works \cite{montagud2012inter} [?, ?, ?] pointed out that delay perception is largely dependent on user differences and context. Thus, we should investigate the noticeable and tolerable boundary of delay, which is statistically suitable for most users. We first refer to a psychology concept called Just Notice Difference (JND) \cite{xu2013exploiting, sat2009statistical}:

\begin{itemize}
    \item \emph{JND}: With other variables fixed, the value for which 50\% of the subjects perceive a difference in their quality.
\end{itemize}

Most related studies recommend using \emph{noticeable delay} and \emph{tolerable delay} as certain values in discussion. We define them as follows:

\begin{itemize}
    \item \emph{Noticeable Delay}: the threshold delay that most users can just perceive. In our experiment, we define it as the 50\% JND of zero delays, i.e., more than 50\% of participants score 1 point for noticeability.
    
    \item \emph{Tolerable Delay}: the threshold delay that most users can just tolerant. In our experiment, we define it as the just intolerable delay minus its 50\% JND, i.e., less than 50\% of participants score 4 or 5 points for disruptiveness.
\end{itemize}

We illustrate the relationship between noticeable delay and tolerable delay in figure xxx.

The noticeable delay is very insightful for network engineering. On the one hand, developers should try to improve the network service, to reach the noticeable delay. On the other hand, when a service is already within noticeable delay, we can appropriately increase the delay to have more room for smoothing or recovering packet loss \cite{xu2013exploiting}.

[NOTE] Tolerable delay indicates a boundary that is nearly intolerant. An application can not simply target at it, because it is already a bad service. Instead, the tolerable delay can be used to assess the Quality of Experience (QoE). [?] suggests a linear correlation between end-to-end delay and user experience. We can use noticeable delay and tolerable delay to determine the correlation.

[NOTE from zsyzgu] Here I change my mind about the tolerant delay. Delay can be perceived by cues for sure, however, the tolerance of delay in a specific task may depend on much more factors such as fairness and interactivity \cite{ishibashi2006subjective, montagud2012inter}. So we are not going to model the tolerant delay anymore, but list factors to affect it according to previous work. We will also introduce a way to measure tolerance of a specific delay in a specific task, in our example experiment.

We next introduce the three synchronization levels. The basic idea is the observation that user perceive delay by cues. We determine the level of a task by judging if it contains cues of \emph{synchronous interaction}, \emph{conversation} and \emph{visual feedback} 这个是按照什么来分的?看关键词感觉不到维度. As Table \ref{tab:table_synchronization_levels} shown, we recommend their noticeable delay and tolerable delay by summarizing previous audiovisual works and adapting to the 3D situation (从audiovisual adapt到3D的策略和原则是什么?).

\begin{table} [!htbp]
\newcommand{\tabincell}[2]{\begin{tabular}{@{}#1@{}}#2\end{tabular}}
\begin{tabular}{|p{0.25\columnwidth}|p{0.2\columnwidth}|p{0.2\columnwidth}|p{0.2\columnwidth}|}
\hline 
\textbf{Levels} and \emph{Examples} & Noticeable Delay & Tolerable Delay & 3.5 MOS \\

\hline
\textbf{Synchronous Interaction} & \textbf{20 - 50 ms} & \textbf{50 - 100 ms} & ?? ms \\
\hline
\emph{Rock-Paper-Scissors} \cite{hashimoto2006influences} & 40 ms & ?? ms & 70 ms \\
\hline
\emph{Virtual car driving} \cite{pantel2002impact} & 50 ms & 200 ms & ?? ms \\
\hline
\emph{example} [?] & ?? ms & ?? ms & ?? ms \\

    
\hline
\textbf{Conversation} & \textbf{100 - 150 ms} & \textbf{300 - 400 ms} & ?? ms \\
\hline
\emph{3D Visual Communication} \cite{wu2009quality} & 120 ms & ?? ms & ?? ms \\
\hline
\emph{Video Group Discussion} \cite{schmitt2014asymmetric} & 500 ms & 1000 ms & 500 ms \\
\hline
\emph{Audiovisual telecommunication} \cite{tam2012video} & ?? ms & ?? ms & 500 ms \\
\hline
\emph{Take turns reading random numbers aloud as quickly as possible} \cite{kitawaki1991sub} & ?? ms & ?? ms & 80 ms \\
\hline
\emph{Take turns verifying random numbers as quickly as possible} \cite{kitawaki1991sub} & ?? ms & ?? ms & 120 ms \\
\hline
\emph{Take turn verifying city names as quickly as possible} \cite{kitawaki1991sub} & ?? ms & ?? ms & 180 ms \\
\hline
\emph{Free conversation} \cite{kitawaki1991sub} & ?? ms & ?? ms & 200 ms \\
\hline
\emph{example} [?] & ?? ms & ?? ms & ?? ms \\

\hline
\textbf{Visual Feedback} & \textbf{150 - 500 ms} & \textbf{500 - 1000 ms} & ?? ms \\
\hline
\emph{example} [?] & ?? ms & ?? ms & ?? ms \\
\hline
\emph{example} [?] & ?? ms & ?? ms & ?? ms \\
\hline
\emph{example} [?] & ?? ms & ?? ms & ?? ms \\
\hline
\emph{example} [?] & ?? ms & ?? ms & ?? ms \\
\hline
\emph{example} [?] & ?? ms & ?? ms & ?? ms \\
\hline
\emph{example} [?] & ?? ms & ?? ms & ?? ms \\
\hline

\end{tabular}
\caption{The three synchronization levels.}
\label{tab:table_synchronization_levels}
\end{table}

[NOTE from LZP] In \cite{schmitt2014asymmetric}, the trial only has three delays, 0ms, 500ms and 1000ms. But it has three kinds of delay, symmetric, moderator and random. Meanwhile, paper[64] is similar to this one.
In \cite{tam2012video}, participants rate the conversation on five-point scale items, but the results are all between 3.5 and 4.5. So I pick the biggest delay as the MOS 3.5 delay.
For the rest of examples in conversation section, I can't summarize them. And in \cite{kitawaki1991sub}, the MOS is in the range of 0 to 4, but I think it is the same as 0 to 5.


\subsection{1. Synchronous Interaction}


\subsubsection{synchronous gesture}

If two distributed users have to gesture exactly at the same time, they may be able to perceive delay like checking their own movement. As \cite{nielsen1993response} explained, 100 ms is an upper boundary for users to fell that the system is running instantaneously. For a better performance, a delay of 30 to 50 ms is needed \cite{chen2007review}.

Imaging that a pair is playing rock-paper-scissors in a 3DTI system with a delay of 100 ms. They expect to perform the gesture exactly at the same time. However, at least one player will find that his partner gesture at least 100 ms slower, which may cause annoyance.

\subsubsection{synchronous speaking}

A pair of users would be sensitive to delay if they have to speak at the same time. For example, when counting down together in a 3DTI system with a delay of 100 ms, at least one user will hear repeated sounds more than 100 ms apart. As a reference, the human ear can distinguish an echo from the original direct sound if the delay is more than 100 ms \cite{wolfel2009distant}. Notice that synchronous speaking is different from a conversation (turn talking).

\subsubsection{instrument ensemble}

With the development of the high-quality network, it is clear that networked music performance has a future \cite{carot2007networked}. 3DTI makes these tasks more nature. For example, two distributed musicians can practice piano duet through 3DTI.

A realistic musical interaction assumes a one-way delay of less than 25 ms \cite{carot2007network}. Beyond this threshold, the groove-building-process cannot be realized by musicians. \cite{schuett2002effects} suggests a delay between 10 - 20 ms for providing a stabilizing effect on the tempo. For a relatively worse network, a coping strategy was discovered that allowed the performers to maintain a solid tempo up to 50 - 70 ms of delay.

\subsection{2. Conversation}

Conversation is an important cue for delay perception. In face-to-face situations, we have learned to unconsciously manage a conversation using the timing of the small pauses in speech \cite{sacks1978simplest}.

[TODO] Theory: Turn Talking Model.

\subsubsection{Chatting}

[TODO]

\subsubsection{Remote guidance}

[TODO]

\subsection{3. Visual Feedback}

[TODO] A short introduction.

[TODO] Theory: Situation Awareness Theory.

[TODO] Theory: Grounding Theory.

\subsubsection{Silent collaboration}

[TODO] Example: Surgery Simulation.

\subsubsection{Imitation}

[TODO] Example: Building Block.

\subsubsection{Turn-based game}

[TODO] Example: Playing chess.

[NOTE] Most actual networks will not exceed such a large delay \cite{donovan2014understanding} [?, ?]

\subsection{[NOTE] Suggestion for network design}

\begin{enumerate}
    \item An application should reach the delay which leads to MOS of 3.5 points.
    \item If already within the noticeable delay, we can appropriately increase the delay to have more room for smoothing and recovering packet loss.
    \item Assistant synchronization can be integrated in an application. For example, we can use synchronized flickers in both side to help a Rock-Paper-Scissors game.
\end{enumerate}

\subsection{[NOTE] Examples of Applications}

\begin{enumerate}
    \item \emph{Rock-paper-scissors}:
    \item \emph{Piano Duet}:
    \item \emph{Chorus}:
    \item \emph{Countdown Together}:
    \item \emph{Chat}:
    \item \emph{Tell-a-lie Game}:
    \item \emph{Building Blocks}:
    \item \emph{Interview}:
    \item \emph{Playing Chess}:
    \item \emph{Building Blocks without Chatting}:
    \item \emph{Magic The Gathering}:
    \item \emph{3D version of Hearthstone}:
    \item \emph{Surgery Simulation}:
    \item \emph{Playing Chess without Seeing Your Partner}:
    \item \emph{Real-time teaching}
    \item \emph{dancing}
\end{enumerate}


\section{System Overview}

% Overview
Our 3DTI system fuses two distributed scenes into a virtual space in full 3D. The end-to-end delay is 50 ms, i.e., the time interval between a user acts and his remote partner sees. The system consists of inexpensive commercial devices (\$ 7000). The project is open-source [?]. Figure xxx illustrates the pipeline of our system.

%Each side: CPU $400; GPU $700; network card $400; RAM $200; Disk $200; HTC Vive $1200; Realsense $150*3.

\subsection{Hardware and Software Overview}

\subsubsection{Hardware}

At each capture site, we had three depth cameras for capturing, a PC for computing and an HMD for rendering. Realsense D415 (depth cameras) were used to capture a volume of $2m \times 2m \times 2m$. The locating place of each camera and its contribution to 3D mesh are illustrated in Figure xx. Each PC had an Intel i7-7700k CPU and a GTX 1080Ti GPU. HTC Vive was used to present the fused reconstruction of both sides. Ten Gigabit network cards (Intel X520-SR2) were used to connect the two capture sites.

\subsubsection{Software}

OpenCV was used for camera calibration. CUDA was used for image processing and the kernel algorithm. Unity3D was used to implement the high-level application. It fetches live reconstruction from the kernel and renders it in HTC Vive.

\subsection{Calibration}

\subsubsection{Calibration between Cameras}

The \emph{camera calibration module} in OpenCV was used to calibrate the cameras. Each pair of cameras took ten snapshots (1080p color images) of a glass-made flat checkerboard. Then, OpenCV aligned their coordinates ($SD < 1 pixel$).

%[from lwq] To achieve a more accurate calibration, we use 1080p color imagine and a glass-made flat checkerboard moving in common area. Then we use the OpenCV library to detect corners in checkerboard. By analyzing a series of frames, we can align coordinates of two cameras. This process can reduce the standard deviation between two cameras to 1 pixel.

\subsubsection{Calibration between HMD and Cameras}

The HTC Vive was calibrated by setting the original point in its software. We placed the original point of the cameras at the same position by using the checkerboard. Hence, we aligned the HTC Vive and the cameras. This calibration is not necessarily accurate because the users can hardly perceive the error [?]. This step also aligned the coordinates of the two capture sites.

%[from lwq] First we use a certain point on table as original point of our virtual space. We also use a black and white checkerboard to unify the coordinates of two terminal. After doing this we can merge space of two users into virtual one and they share one common table, sitting in each side of it. Then we put the HTC VIVE on the original point and facing certain direction to calibrate the VR glass and virtual world coordinate. 

\subsection{Preprocessing}

\subsubsection{Depth Processing}

The cameras acquired depth images of $640 \times 480$ pixels at 30 FPS. The Realsense D415 is based on binocular disparity. Thus, disparity values (instead of depth values) were used in the processing for accuracy. We applied median filtering, spatial filtering, hole filling and temporary filtering on the depth images.

%We use CUDA to reimplement the depth processing module in RealSense SDK for efficiency. The RealSense acquires depth images with $640 \times 480$ pixel resolution at 30 FPS. RealSense D400-series are based on binocular disparity. It is more accurate to filter depth information in disparity values. We upload origin images in disparity values from CPU to GPU. Then we apply median filtering, spatial filtering, hole filling and temporary filtering on them. Finally, we translate the disparity values into depth images.

\subsubsection{Color Processing}

% We upload the original images in YUV422 format from CPU to GPU. Finally, we translate images into the RGB format in GPU.

The cameras acquired color images of $960 \times 540$ pixels at 30 FPS. We manually adjusted the explosion setting of the RealSense. We used one RGB camera as a reference and warped the other cameras to this reference by white balancing and linear mapping.

\subsubsection{Background Removal}

The system fuses physical worlds in both sides into a virtual scene. It should remove unnecessary background and retain only the shared objects and the individuals. We recorded RGBD images as background in the calibration step. At runtime, we removed pixels that are similar to the background based on thresholds.

\subsection{3D Reconstruction}

[from lwq]We divide the reality space into $256 \times 256 \times 256$ voxels and calculate SDF values using point clouds provided by all the depth cameras parallelly on GPUs. By doing that we can merge these point clouds. When calculating the SDF value, we use a weighted average instead of simple average where  $W_{i} = \frac{1}{Dist}$, considering the error of Realsense Depth Camera is proportional to the distance from the camera to the object. This process can improve the quality of mesh and reduce some noise from hardware. 
    
As we need to merge point clouds from two clients, we need to retain some objects from the other depth camera array. In one of our application, we fuse two chessboards along with half of the chess each side together to implement remote gaming. In this case, we need to retain the chess from both sides. So we run TSDF on each side separately and take the minimum of SDF values to gain the actual surface of the merged mesh. A simple example of this procession is shown in Figure XX.

\begin{figure*}[!htbp]
\centering
\includegraphics[width=14cm,height=4cm]{figures/figure_tsdf.png}
\setlength{\abovecaptionskip}{-0.5cm}
\caption{Left to right: 1)SDF value from camera 1; 2)Merged SDF value from camera 2; 3)SDF value by simple average; 4)Merged SDF value from our project.}
\label{3}
\end{figure*}
    
Then we use the marching cube to convert the SDF value of each volume to a triangle mesh, which can be done parallelly on GPUs. To fully make use of color images from cameras, we divide one triangle into four triangles by midpoints of each edge. Then we can match color imagine with these triangles for a better texture quality without a large effect on efficiency(Figure YY).

\begin{figure}[H]
\centering
\includegraphics[width=6cm,height=3cm]{figures/figure_mc.png}
\setlength{\abovecaptionskip}{0.5cm}
\caption{Left: Mesh without Supersampling; Right: Mesh witht Supersampling.}
\label{4}
\end{figure}


\subsection{Delay Control}
[from lwq]To gain a more credible result of delay preception in 3DTI, we need to have accurate control over our system. First, we synchronize the depth camera array using Realsense SDK. We set the frame rate of cameras at 30FPS and also regard it as the system clock. After that, we can start our procession when depth cameras obtain new data. The latency of our system is controllable, and it can be minimized to XXXms. Figure XXX shows the pipeline and estimation of the delay of our system when using three depth cameras on each side. To have better control on the delay, we use a buffer to store the data transmitted from the remote client. We can alter the latency by changing the time each set of data remains in the cache.

\begin{figure*}[!htbp]
\centering
\includegraphics[width=14cm,height=7cm]{figures/figure_pipeline.png}
\setlength{\abovecaptionskip}{0.5cm}
\caption{}
\label{5}
\end{figure*}


\section{User Experiment}

[Yu] The goal of the experiment is ...

The experiment has two parts. Part A is a chess game with different conditions of cues. Results support the idea of classification in our framework. Part B is the Rock-Paper-Scissors game. It requires a low delay. We present a simple assistant design to help synchronization.

The experiment is also an example. It illustrates how to measure the noticeable delay and the acceptable delay for a specific application.

\subsection{Part A: Playing Chess}

Part A was to study the variety of delay perception in different synchronization level. The task was a chess game between pairs of participants in two rooms, with two conditions of cues: audiovisual mode and visual only mode.

\subsubsection{Experimental design}

We used a within-subjects experimental design. Each pair of participants played chess in two sessions of Communication Channel (CC): \emph{audiovisual CC} and \emph{visual only CC}, which correspond to the 2nd and 3rd synchronization level. The CC conditions were assigned to participant pairs in a Latin square design. Delay was another within-subjects factor with five conditions (50, 150, 250, 450, 750 ms in \emph{audiovisual CC}; 150, 450, 1050, 1550, 2050 ms in \emph{visual only CC}). In each session, we tested the five delay conditions in five trials. The delay conditions were assigned in a random order. In particular, we encouraged participants to chat in the \emph{audiovisual CC}.

\subsubsection{Task}

In each trial, two distributed participants played chess "face-to-face" for three minutes. We adopted \emph{Reversi} as the task. \emph{Reversi} is simple enough that the participants can learn it in a short time. The game involves frequent interaction: when capturing, a participant should ask his partner to remove the captured chess. The participants have enough chances to perceive a noticeable delay.

In the physical world, each player interacted with a chessboard and chess pieces on his own side. The 3DTI system fused the two physical scenes into a virtual space. In the virtual space, each player could see not only his chess pieces but also his partner' s chess pieces.

\subsection{Part B: Rock-Paper-Scissors}

Part B was to evaluate the impact of delay in a high delay requirement situation. We used the Rock-Paper-Scissors game. We designed a synchronized audio cue to help users to synchronize with each other. The study assesses its effect on user experience.

\subsubsection{Experimental design}

This part was also a within-subjects design. Each pair played the Rock-Paper-Scissors game in two sessions: with and without the synchronized audio cue. We applied Latin square to the two sessions. In each session, we tested the delay of 50, 83, 117 and 150ms in four trials. The order of delay conditions was random. We also adopted Latin square on part A and part B.

\subsubsection{Task}

In each trial, a pair continuously drew the Rock-Paper-Scissors gestures until one of them won for ten times. There were two conditions to test: with and without the synchronized visual cue. In an actual network, it is possible to synchronize the time of two systems with almost zero milliseconds apart (the NTP protocol [?]).

Our 3DTI system provided zero-delay audio cues for both users to help them gesture exactly at the same time. The cue was an audio source of four seconds, with "tick" sounds at the 2nd, 3rd second and a "tack" sound at the 4th second. We told participants to gesture when their hear the "tack" sound.

\subsection{Participants}

We advertised our experiment on social media. Sixteen pairs of participants took part in our experiment (32 in total, xx females). They all came from the campus, aged from xx to xx. Participants were paid 150 yuan for the 90 minutes long study. The ten participants with the most conversation turn received extra 50 yuan.

Previous works have pointed out that the individual user differences affect study results of delay perception [?]. In our experiment, we control the source of participants carefully:

\begin{itemize}
    \item \emph{Relationship}: Each pair of participants are familiar with each other (friends, classmates or partners). This setting is to improve the conversation quality.
    
    \item \emph{First language}: All the participants are native Chinese speakers. Chinese conversations are a little bit harder to predict compared to English conversations [?], which may lead to a larger noticeable delay (about xx ms).
    
    \item \emph{Experience in DIME}: Our participants have relatively high education levels. According to the self-report questionnaire, they are quite familiar with audiovisual multimedia (xx points in average, 5 for experts) and AR/VR (xx points in average).
\end{itemize}

Thus, our study results are rigorous but relatively low in the external validity. We recommend a larger amount of participants if the readers need a more general result.

\subsection{Procedure}

%[Copy from other paper] Each pair of participants completed consent forms at the study location. They were then taken to separate study rooms containing the audiovisual telecommunications stations. The experimenters informed the participants that they would have seven 4-minute conversations using the stations, and that they would be given topic sheets for inspiration. Participants could use a small timer to keep track of their conversation, and they were informed that the experimenter would interrupt the conversation once four minutes had passed. Once seated, participants were given headphones and the first topic sheet. Participants were told to start whenever they both were ready. The experimenters then left the study rooms to monitor the conversations from a nearby location.

%[Copy from other paper] After four minutes, the experimenters interrupted the conversations, gave the participants short surveys to complete, and presented the next topic sheet. This process was repeated for each of the seven trials. After all seven trials, participants completed a questionnaire asking about their favorite conversations and any difficulties with understanding the other participant.

Before the experiment, we invited the participant pair to a room and explained our study. We explained the rule of \emph{Reversi} and \emph{Rock-Paper-Scissors}. Next, the pair had ten minutes to experience the physical interaction of these two games. We asked the participants to remember the feeling of physical interaction and regard it as a zero-delay experience. Then, we introduced our experimental procedure to the participants.

Part A had $2 sessions \times 5 trials = 10 trials$, which lasted for 40 minutes. Part B had $2 sessions \times 4 trials = 8 trials$ (20 minutes). In each trial, the participants experience the remote VR game. After each trial, participants filled in a short survey and rested for one minute. The questions in the survey are shown in Table \ref{tab:table_experiment}.

\begin{table} [!htbp]
\begin{tabular}{|p{0.25\columnwidth}|p{0.35\columnwidth}|p{0.3\columnwidth}|}
\hline 
Label & Question & Scale \\
\hline
quality & How do you feel during the experiment? & Bad <--> Excellent \\
\hline
noticeability & Can you perceive the delay in the connection? & Very much <--> Not at all \\
\hline
tolerance & To what extent where you annoyed by the delay? & Severe annoyance <--> No annoyance \\
\hline
\end{tabular}
\caption{Questions and scale.}
\label{tab:table_experiment}
\end{table}

After each session, participants had a five minutes break. We conducted brief interviews with some subjective questions as followed:

\begin{itemize}
    \item How do you notice the delay? What are the cues?
    
    \item What makes you annoyed in the task?
    
    \item Any other comments?
\end{itemize}

In particular, we explained to participants about the concept of network delay. In our system, there was a local end-to-end delay of about 20 ms, which is slightly noticeable for the users. We told the participants that we do not consider the local delay in the questionnaire. Instead, we evaluate the network delay, which is the time interval between your partner acts and you see. A participant can perceive the network delay by judging if his partner is slow in speaking and gesturing.

\subsection{Results}


[NOTE from zsyzgu] I have several ideas about the result:

\begin{itemize}
    \item In 3DTI, the users are much more tolerable than in audiovisual DIME. The immersive environment makes users focus on the interaction itself.
    \item A simple synchronized audio cue can help synchronization in tasks with high network requirement.
    \item Conversation is a stronger cue compared with visual feedback.
    \item Delay perception of 50 ms and 150 ms in audiovisual model have no significant difference.
    \item The individual user differences are larger in 3DTI.
\end{itemize}


\subsubsection{Delay Effects}[TODO]

Attention: Analysis from only 6 groups of data.

\begin{figure}[H]
\centering
\includegraphics[width=6cm]{figures/figure_experiment1_NQ.jpg}
\setlength{\abovecaptionskip}{0.5cm}
\caption{Effects of delay on quality score in part A(Playing Chess).}
\label{4}
\end{figure}

\begin{figure}[H]
\centering
\includegraphics[width=6cm]{figures/figure_experiment1_DP.jpg}
\setlength{\abovecaptionskip}{0.5cm}
\caption{Effects of delay on noticeability score in part A(Playing Chess).}
\label{4}
\end{figure}

\begin{figure}[H]
\centering
\includegraphics[width=6.9cm]{figures/figure_experiment2_NQ.jpg}
\setlength{\abovecaptionskip}{0.5cm}
\caption{Effects of delay on quality score in part B(Rock-Paper-Scissors).}
\label{4}
\end{figure}

\begin{figure}[H]
\centering
\includegraphics[width=6.9cm]{figures/figure_experiment2_DP.jpg}
\setlength{\abovecaptionskip}{0.5cm}
\caption{Effects of delay on noticeability score in part B(Rock-Paper-Scissors).}
\label{4}
\end{figure}


\subsubsection{Data Analysis}[TODO]

\begin{table} [!htbp]
\begin{tabular}{|p{0.3\columnwidth}|p{0.15\columnwidth}|p{0.3\columnwidth}|p{0.15\columnwidth}|}
\hline 
Difference & 0.30000 & t Radio & 0.582772 \\
\hline
Std Err Dif & 0.5148 & DF & 15.92888 \\
\hline
Upper CL Dif & 1.3917 & Prob > |t| & 0.5682 \\
\hline
Lower CL Dif & -0.7917 & Prob > t & 0.2841 \\
\hline
Confidence & 0.95 & Prob < t & 0.7159 \\
\hline
\end{tabular}
\caption{Oneway analysis of noticeability score by mode(Delay = 150ms)}
\label{tab:table_questionnaire}
\end{table}


\begin{table} [!htbp]
\begin{tabular}{|p{0.3\columnwidth}|p{0.15\columnwidth}|p{0.3\columnwidth}|p{0.15\columnwidth}|}
\hline 
Difference & 1.0000 & t Radio & 1.678363 \\
\hline
Std Err Dif & 0.5958 & DF & 14.9258 \\
\hline
Upper CL Dif & 2.2705 & Prob > |t| & 0.1141 \\
\hline
Lower CL Dif & -0.2705 & Prob > t & 0.0570 \\
\hline
Confidence & 0.95 & Prob < t & 0.9430 \\
\hline
\end{tabular}
\caption{Oneway analysis of noticeability score by mode(Delay = 450ms)}
\label{tab:table_questionnaire}
\end{table}


\section{Limitation and FUTURE WORK}

Our work has some limitations. First, the rendering quality of our system is not state-of-the-art. Our pipeline wastes a portion of time in idling. There is still a space to improve the quality while maintaining the high synchronicity. Second, while co-presence has the potential to support eye contact, the head-mounted display gets in the way of eye contact. In the future, it is possible to reconstruct users' eyes by using inward cameras of HMDs. Third, we did not focus on the transmission so far. Efficient compression should be developed to support the 3DTI system in an acutal network.


\section{Conclusion}

In this paper, we demonstrate TeleCP, a software framework for 3D telepresence that adequately supports co-presence. The framework bases on some commercial hardware (Realsense, PCs and HTC Vive, \$7000 in total). It allows a high level of co-presence that distributed users can communicate and interact with each other in the same virtual space at the same time. TeleCP has three new features compared to prior work. First, the one-way end-to-end delay is as low as 50 ms. Second, it provides an optional sync assistance to help the users perform at the same time regardless of the unavoidable network delay. Third, TeleCP to some extent provides shared props. The experiment found that these features improve the user experiences and allow some natural interactions that could only happen in physical communications before.

\bibliographystyle{ACM-Reference-Format}
\bibliography{bibliography}

\end{CJK*}

\end{document}
