\section{Related Work}

% [paragraph] related work structure
In this section, we first review 3D tele-immersion techniques. We summarize the necessary components of current 3DTI systems to guide our implementation. Then, we have a look about existing studies on delay in 3DTI. Much more related researches were conducted in audiovisual DIME (Distributed Interactive Multimedia Environments). We discuss them later to help forming our framework.

\subsection{3D Tele-immersion}

% [paragraph] introduction of 3D Tele-immersion
Optimal techniques toward 3DTI became clear in the last decade. Basically, a 3DTI system requires three processes: reconstruction, transmission and rendering \cite{fuchs2014immersive}. For 3D reconstruction, the volumetric algorithm has become mainstream. We applied TSDF Volume \cite{curless1996volumetric} and Marching Cubes \cite{lorensen1987marching} to fuse depth images into a polygonal mesh. We do not focus on transmission as \cite{beck2013immersive, pece2011adapting} did, but use 10 Gigabit optical fiber between computers instead. For rendering, we applied the head-mounted display (HTC Vive) because of its on-going growth. In this subsection, we review previous works of 3D reconstruction and rendering in details.

\subsubsection{3D Reconstrucion}

% [paragraph] cameras sea (<2000)
In early works, researchers used an array of cameras to capture dynamic scenes \cite{kanade1997virtualized, fuchs1994virtual}. For a given camera view, these systems create a polygonal model that will look correct. They do not actually construct a 3D model.

% [paragraph] toward volumetric reconstruction 2000 ~ 2010
TELEPORT \cite{gibbs1999teleport} composites video-textured surfaces within 3D geometric models. It uses only one camera. In 2002, researchers started to design virtual 3D environment with multiple cameras \cite{gross2003blue, towles20023d}. However, their 3D reconstruction result was only point cloud. In 2008, Kurillo et al. presented a framework for remote collaboration and training of physical activities \cite{kurillo2008immersive}. This work tried a reconstruction method with triangulation, but only reached the frame rate of about 5-7 FPS. \cite{loop2013real} and \cite{petit2010multicamera} for the first time presented compelling real-time reconstruction techniques with multiple cameras. However, the lack of depth dimension indicated their modeling with only silhouette boundaries.

% [paragraph] toward volumetric reconstruction 2010 ~ 2015
Researchers achieved the real-time performance of high-quality reconstruction in the last decade. In October 2011, Maimone et al. presented a 3DTI system with Kinects \cite{maimone2011encumbrance}. They developed a pixel-based mesh generation algorithm and reached a frame rate of 30 FPS. This work was followed by Beck et al.'s group-to-group telepresence system \cite{beck2013immersive}. In the same month, however, Microsoft introduced KinectFusion \cite{izadi2011kinectfusion} based on volumetric method. They described a novel GPU-based pipeline and achieved a better reconstruction quality. In the next year (2012), Maimone et al. also turned to volumetric methods \cite{maimone2012real} to improve the quality. A huge amount of works improved 3D reconstruction within the same framework as KinectFusion, in the region of scale \cite{niessner2013real, chen2013scalable}, noise reduction \cite{khoshelham2012accuracy, nguyen2012modeling, newcombe2015dynamicfusion} and so on.

% [paragraph] toward fusion4D and its drawback (2016)
In 2016, Microsoft proposed a new pipeline named Fusion4D \cite{dou2016fusion4d}, which is highly robust to occlusions, large frame-to-frame motions, and topology changes. "The fourth dimension" is the time dimension, indicating that it leverages temporally coherence of physical scenes. In the same year, Microsoft integrated fusion4D into their 3DTI system Holoportation \cite{orts2016holoportation}. However, Fusion4D is extremely complex and not open-source. Even with costly devices, Holoportation has an end-to-end latency of 80ms, which can not be ignored in our study. In this paper, we apply a 3D reconstruction method similar to \cite{maimone2012real} (2012) for responsiveness.

\subsubsection{3D Rendering}

% [paragraph] categories of 3D rendering, end light field displays here
Rendering techniques in 3DTI systems can be mainly divided into three categories: light field displays, spatially immersive displays (SIDs) and head-mounted displays (HMDs). The light field displays \cite{jones2007rendering, jurik2011prototyping, kim2012telehuman, gotsch2018telehuman2} suffers from low resolution because neither computing nor rendering devices can support high-quality 4d light fields. SIDs were earlier, while HMDs are becoming popular nowadays.

% [paragraph] spatially immersive displays - earlier
Around year 2000, SIDs had become increasing significant \cite{gross2003blue}. CAVE \cite{cruz1993surround} is a typical SIDs system, which consists of surround-screen projection. Users wear 3D glasses in a CAVE. Most 3DTI systems at that time applied rendering techniques similar to CAVE \cite{gibbs1999teleport, towles20023d, gross2003blue, kurillo2008immersive, benko2012miragetable}. CAVE was design to support the one-to-many presentation. Latter researchers improved it for multi-user by using polarization or time-sharing \cite{frohlich2005implementing, kulik2011c1x6, guan2018two}. Multi-user SID was used by an immersive group-to-group telepresence \cite{beck2013immersive}. There is also a simplified technique called head-tracked auto-stereo display \cite{benko2014dyadic, jones2014roomalive}, which allows 3D view without glasses. Some 3DTI system \cite{maimone2011encumbrance, maimone2012real, pejsa2016room2room} used it for rendering. However, these glasses-free systems have to abandon the benefit of stereoscopy.

% [paragraph] head-mounted displays - popular now
Recently, HMDs are becoming popular. More 3DTI systems tend to apply HMDs for 3D rendering \cite{orts2016holoportation, maimone2013general, lindlbauer2018remixed, smith2018communication}. HMDs are basically cheaper and easier to deploy compared to SIDs. Another superiority of HMDs is their ability to support co-located collaboration \cite{maimone2013general, orts2016holoportation}, i.e., users feel like exactly in the same place. For comparison, SIDs do not support rendering in full 3D, with which a 'window' separate users into two virtual spaces. In 2018, Microsoft proposed Remixed Reality \cite{lindlbauer2018remixed}. This approach combines the benefits of augmented reality and virtual reality using 3D reconstruction and VR HMD. Users can not only see their environment but can also apply changes to it. Finally, we applied the head-mounted VR (HTC Vive) for rendering.

\subsection{Delay Perception in 3DTI}

% [paragraph] introduction of delay perception in telepresence
User experience often relates to QoS (quality of service) including delay, bandwidth, jitter and packet loss \cite{donovan2014understanding}. Previous works have found that delay is one of the most crucial factors determining user experience in telepresence \cite{vogel1995distributed, brunnstrom2013qualinet, schmitt2014asymmetric, schmitt2013qoe}. For telephone, 150ms has been established as an industry standard for an acceptable delay \cite{rec2003g, percy1999understanding}. Also, a huge amount of related works have been conducted in audiovisual DIME. 

% [paragraph] what is different in 3D?
3DTI is quite different from audiovisual DIME: first, high level of immersion offers more cues, i.e., users may be more sensitive to delay in 3DTI; second, with abundant sensory stimuli, users are more tolerant to delay \cite{tam2012video}; third, 3DTI can support much more possible applications that we have to discuss them case by case. Thus, we have to rebuild the theoretical framework of delay perception in 3DTI.

% [paragraph] existing work of delay perception in 3DTI
In the area of 3DTI, most works focus on algorithm and pipeline. Negative impacts of large delay are widely reported \cite{beck2013immersive, gibbs1999teleport, maimone2011encumbrance, kurillo2008immersive, raghuraman2015distortion}. However, only a few works were conducted to study delay perception in 3DTI \cite{wu2009quality, wu2010m, huang2012towards}. These works do not exactly focus on delay perception. Moreover, they are limited by single scenario and immature techniques, e.g., the 2D screen was used to display 3D scenes. In this paper, we explore delay perception in a full 3D tele-immersion. We consider various scenarios and finally form a framework to understand this problem.
