\section{Literature Review and Background}

3D telepresence requires three processes: reconstruction, transmission and rendering \cite{fuchs2014immersive}. In this section, we first review 3D reconstruction techniques. Parallel computing becomes important for a real-time high-quality reconstruction. Then, we discuss the transmission requirement of network delay by reviewing delay perception in tele-communication. Last, we analyze the sense of immersion supported by different display devices. We recommend Head-Mounted Display (HMD) as a practical approach to support a high level of immersion.

%We do not focus on transmission techniques as \cite{beck2013immersive, pece2011adapting} did, but use a 10 Gigabit Ethernet connection

% We used TSDF Volume \cite{curless1996volumetric} and Marching Cubes \cite{lorensen1987marching} for reconstruction, a 10 Gigabit Ethernet connection for transmission, and Unity3D for rendering in HTC Vive.

% A 3DTI system requires three processes: reconstruction, transmission and rendering \cite{fuchs2014immersive}. For reconstruction, volumetric methods have become mainstream, e.g., Truncated Signed Distance Function (TSDF) Volume \cite{curless1996volumetric}, Marching Cubes \cite{lorensen1987marching} and Fusion4D \cite{dou2016fusion4d}. Our system does not focus on transmission as \cite{beck2013immersive, pece2011adapting} did, but uses a direct Ethernet connection instead. For rendering, we recommend Head-Mounted Display (HMD) because it supports co-presence.

\subsection{3D Reconstruction}

In the late 20th century, 3D reconstruction was either an off-line concept \cite{lorensen1987marching, curless1996volumetric} or simple polygonal models that look correct \cite{kanade1997virtualized, fuchs1994virtual, gibbs1999teleport}. In the early 21st century, researchers achieved quasi real-time methods based on point cloud \cite{gross2003blue, towles20023d} and triangulation \cite{kurillo2008immersive, petit2010multicamera, maimone2011encumbrance}.

In the past decade, parallel computing devices (e.g., GPUs) became powerful. 3D reconstruction reached the real-time performance \cite{maimone2012real, loop2013real} through volumetric methods that work in parallel. Microsoft's KinectFusion \cite{izadi2011kinectfusion} is a representative work. They described a GPU-based pipeline that generates high-quality 3D models in real time. Much previous work focused on improving 3D reconstruction within the framework of KinectFusion in regions of scale \cite{niessner2013real, chen2013scalable}, noise reduction \cite{khoshelham2012accuracy, nguyen2012modeling, newcombe2015dynamicfusion} and so on.

% [paragraph] toward fusion4D and its drawback (2016)
In 2016, Microsoft proposed the state-of-the-art pipeline named Fusion4D \cite{dou2016fusion4d}, which is highly robust to occlusions, large frame-to-frame motions, and topology changes. The fourth dimension is time, indicating that it leverages inter prediction. In the same year, Microsoft integrated fusion4D into their 3D telepresence system Holoportation \cite{orts2016holoportation}. However, Holoportation is expensive, not open-source and not responsive enough. Our framework applies a 3D reconstruction method similar to KinectFusion \cite{izadi2011kinectfusion}, which is based on Truncated Signed Distance Function (TSDF) Volume \cite{curless1996volumetric} and Marching Cubes \cite{lorensen1987marching}.

\subsection{Delay perception in tele-communication}

Network delay is a crucial factor that affects user experience in tele-communication \cite{brunnstrom2013qualinet, schmitt2013qoe, wu2009quality}. Numerous studies have been carried out to explore delay perception in telephone and 2D tele-communication.

%The noticeability of network delay and the overall rating of network quality are two important factors measured by most studies \cite{geerts2011we, schmitt2014asymmetric, schmitt2014influence, wu2009quality}.

For audio-medicated communication, one-way delay of 150 ms has become a standard for good user experience \cite{recommendation2003114}. \cite{gergle2006impact} summarized prior works on audio delay and found that delays below 300 ms pose little problem, while delays above 450 ms can severely impact communication.

For 2D tele-communication, \cite{tam2012video, polycom2006traffic} reviewed prior works and found that one-way delays would be noticeable between 100 and 150 ms. Compared to the audio-medicated communication, delay had a weaker impact on naturalness when both audio and video channels were available \cite{tam2012video}.

% [paragraph] existing work of delay perception in 3DTI
For 3D telepresence, negative impacts of large network delay are widely reported \cite{beck2013immersive, gibbs1999teleport, maimone2011encumbrance, kurillo2008immersive, raghuraman2015distortion}. A few works studied delay perception in 3D telepresence \cite{wu2010m, huang2012towards, wu2009quality}. \cite{wu2009quality} conducted a study in a simple 3D telepresence system that renders live reconstruction in 2D screen. They found that an one-way network delay of 120 ms can be predictable and disruptive in their system. To our knowledge, no work has been done to study delay in 3D telepresence with a high level of co-presence. It is unknown whether the standard of 100 ms to 150 ms is enough in 3D.

\subsection{3D Rendering}

Human sense the 3D environment by four types of cues: pictorial depth cues, motion parallax, binocular vision and accommodation \cite{kooi2004visual}. Different display devices support different levels of 3D perception. For examples, 2D screens provide pictorial depth cues; head-tracked auto-stereo displays \cite{maimone2011encumbrance, maimone2012real, pejsa2016room2room} support pictorial depth cues and motion parallax.

Light field displays \cite{jones2007rendering, jurik2011prototyping, kim2012telehuman, gotsch2018telehuman2} can fully support the sense of 3D. However, they suffer from low resolution because of the huge computation. Most practical 3D displays support pictorial depth cues, motion parallax and binocular vision, but not accommodation. Spatially Immersive Displays (SIDs) and Head-Mounted Displays (HMDs) are the two typical techniques.

% Pictorial depth cues can provide 3D perception with one eye, .e.g., cast shadows and occlusion. Motion parallax is that the user can change his point of view and direction of gaze naturally in a 3D environment \cite{fuchs2014immersive}. Binocular vision is that human can perceive depth by the slightly different location of the left and right eyes. Accommodation is the automatic adjustment of the focus of the eye. Table xxx shows how existing display devices support these cues.

Around year 2000, SIDs had become increasing significant \cite{gross2003blue}. CAVE \cite{cruz1993surround} is a typical SID system that consists of surround-screen projection. Most 3D telepresence systems at that time applied rendering techniques similar to CAVE \cite{gibbs1999teleport, towles20023d, kurillo2008immersive, benko2012miragetable}. Latter researchers improved SIDs by supporting multi-user \cite{frohlich2005implementing, kulik2011c1x6, guan2018two} and glasses-free experience \cite{benko2014dyadic, jones2014roomalive, pejsa2016room2room}.

% However, SIDs can not support co-presence because there is an unavoidable 'window' that separates users into two virtual spaces.

Recently, HMDs become popular. More 3D telepresence systems tend to apply HMDs for 3D rendering \cite{orts2016holoportation, maimone2013general, lindlbauer2018remixed, smith2018communication}. HMDs are basically cheaper and easier to deploy compared to SIDs. There are two categories of HMDs: Virtual Reality (VR) and Augmented Reality (AR). Remixed Reality \cite{lindlbauer2018remixed} suggests that we can leverage both the benefits of AR and VR by rendering live 3D reconstruction in VR.

% Another superiority of HMDs is the ability to support co-presence \cite{maimone2013general, orts2016holoportation}.

%\subsection{Media Richness in 3DTI}

%Media richness theory suggests that communication with rich media information can help reduce uncertainty and equivocality \cite{daft1986organizational}. Media richness is defined as the potential information carrying capacity of data. Different media have varied richness and can be determined by three factors: immediate feedback capacity, multiple cues and language restriction \cite{daft1983information}. Face-to-face communication is the richest form of information processing. It provides immediate feedback, contains multiple cues such as audio and visual information and is the most natural way of human communication. 

%The theory can be applied to 3DTI study as well. Previous work in building 3DTI system focused mainly on rendering quality of visual space \cite{kurillo2008framework, kurillo2008immersive}.  However, we argue that in addition to vision, there are also other important factors that contribute to good quality of user experience, such as audio, tactility and co-presence. Audio and tactile sense extends the potential information carried by data transmission, which users can use to reduce uncertainty and correct misunderstanding instantly. Meanwhile, co-presence enables two users feel like they were having a face-to-face communication and would probably improve quality of user experience. We believe a good 3DTI system should be developed upon a rich set of media information with all these factors taken into consideration.

