\section{A Conceptual Framework of Delay Perception in 3DTI}

% The first level \emph{Synchronous} tasks involves interactions at the very same time, e.g., instrument ensemble, dancing together and the Rock-Paper-Scissors game. They require the lowest network delay. We suggest that 3DTI supports much more tasks in this delay-sensitive level compared to the 2D situation, because some of these tasks are impossible or unnatural in 2D.

%We present a conceptual framework of delay perception in 3DTI. \textbf{In this framework, 3DTI applications are divided into three levels according to their delay requirement. 这句话不清楚,很想知道什么是一个level,但是没有满足预期,猜也猜不到}. Suggestions are given to network engineering for each level. \textbf{Furthermore, we have designed an experiment to illustrate the framework.} Readers can reconstruct the experiment to precisely measure the delay requirement of a specific application.

% The structure of our framework
We present a conceptual framework of delay perception in 3DTI. The framework divides 3DTI tasks into three levels: tasks with \emph{synchronous interaction}, \emph{conversation} and \emph{only visual feedback}. Correspondingly, the three levels of tasks call for high, middle and low requirement of network delay. For each level, the framework also gives suggestions to improve the perceived quality of network.

The framework is supported a comprehensive reviews of previous works. Delay perception is a well-understood research question in audiovisual DIME. \textbf{Our framework partly relies on previous theories and study results(这句话不清楚)}. However, there is a large difference of delay perception between 3DTI and audiovisual DIME. Thus, we take the features of 3DTI into account in order to form our framework.

Noticeability and tolerance of delay are two important factors that were measured by most studies \cite{wu2009quality, schmitt2014influence, geerts2011we, schmitt2014asymmetric}. Some studies also focus on users' perception of overall network quality. These metrics are users' subjective rating in a specific task, and with a specific delay. In our study, we assess subjective feedback via questionnaires similar to \cite{schmitt2014influence}. The questionnaire is on a 5-point Likert scale (Table \ref{tab:table_questionnaire}).

\begin{table} [!htbp]
\begin{tabular}{|p{0.25\columnwidth}|p{0.35\columnwidth}|p{0.3\columnwidth}|}
\hline 
Label & Question & Scale \\
\hline
quality & How do you feel during the experiment? &Bad <--> Excellent \\
\hline
noticeability & Can you perceive the delay in the connection? & Very much <--> Not at all \\
\hline
tolerance & To what extent where you annoyed by the delay? & Severe annoyance <--> No annoyance \\
\hline
\end{tabular}
\caption{Questions and scale.}
\label{tab:table_questionnaire}
\end{table}

A 3DTI developer may expect a certain recommended delay for his application. However, previous works \cite{montagud2012inter} [?, ?, ?] pointed out that delay perception is largely dependent on user differences and context. Thus, we should investigate the noticeable and tolerable boundary of delay, which is statistically suitable for most users. We first refer to a psychology concept called Just Notice Difference (JND) \cite{xu2013exploiting, sat2009statistical}:

\begin{itemize}
    \item \emph{JND}: With other variables fixed, the value for which 50\% of the subjects perceive a difference in their quality.
\end{itemize}

Most related studies recommend using \emph{noticeable delay} and \emph{tolerable delay} as certain values in discussion. We define them as follows:

\begin{itemize}
    \item \emph{Noticeable Delay}: the threshold delay that most users can just perceive. In our experiment, we define it as the 50\% JND of zero delays, i.e., more than 50\% of participants score 1 point for noticeability.
    
    \item \emph{Tolerable Delay}: the threshold delay that most users can just tolerant. In our experiment, we define it as the just intolerable delay minus its 50\% JND, i.e., less than 50\% of participants score 4 or 5 points for disruptiveness.
\end{itemize}

We illustrate the relationship between noticeable delay and tolerable delay in figure xxx.

The noticeable delay is very insightful for network engineering. On the one hand, developers should try to improve the network service, to reach the noticeable delay. On the other hand, when a service is already within noticeable delay, we can appropriately increase the delay to have more room for smoothing or recovering packet loss \cite{xu2013exploiting}.

[NOTE] Tolerable delay indicates a boundary that is nearly intolerant. An application can not simply target at it, because it is already a bad service. Instead, the tolerable delay can be used to assess the Quality of Experience (QoE). [?] suggests a linear correlation between end-to-end delay and user experience. We can use noticeable delay and tolerable delay to determine the correlation.

[NOTE from zsyzgu] Here I change my mind about the tolerant delay. Delay can be perceived by cues for sure, however, the tolerance of delay in a specific task may depend on much more factors such as fairness and interactivity \cite{ishibashi2006subjective, montagud2012inter}. So we are not going to model the tolerant delay anymore, but list factors to affect it according to previous work. We will also introduce a way to measure tolerance of a specific delay in a specific task, in our example experiment.

We next introduce the three synchronization levels. The basic idea is the observation that user perceive delay by cues. We determine the level of a task by judging if it contains cues of \emph{synchronous interaction}, \emph{conversation} and \emph{visual feedback} 这个是按照什么来分的?看关键词感觉不到维度. As Table \ref{tab:table_synchronization_levels} shown, we recommend their noticeable delay and tolerable delay by summarizing previous audiovisual works and adapting to the 3D situation (从audiovisual adapt到3D的策略和原则是什么?).

\begin{table} [!htbp]
\newcommand{\tabincell}[2]{\begin{tabular}{@{}#1@{}}#2\end{tabular}}
\begin{tabular}{|p{0.25\columnwidth}|p{0.2\columnwidth}|p{0.2\columnwidth}|p{0.2\columnwidth}|}
\hline 
\textbf{Levels} and \emph{Examples} & Noticeable Delay & Tolerable Delay & 3.5 MOS \\

\hline
\textbf{Synchronous Interaction} & \textbf{20 - 50 ms} & \textbf{50 - 100 ms} & ?? ms \\
\hline
\emph{Rock-Paper-Scissors} \cite{hashimoto2006influences} & 40 ms & ?? ms & 70 ms \\
\hline
\emph{Virtual car driving} \cite{pantel2002impact} & 50 ms & 200 ms & ?? ms \\
\hline
\emph{example} [?] & ?? ms & ?? ms & ?? ms \\

    
\hline
\textbf{Conversation} & \textbf{100 - 150 ms} & \textbf{300 - 400 ms} & ?? ms \\
\hline
\emph{3D Visual Communication} \cite{wu2009quality} & 120 ms & ?? ms & ?? ms \\
\hline
\emph{Video Group Discussion} \cite{schmitt2014asymmetric} & 500 ms & 1000 ms & 500 ms \\
\hline
\emph{Audiovisual telecommunication} \cite{tam2012video} & ?? ms & ?? ms & 500 ms \\
\hline
\emph{Take turns reading random numbers aloud as quickly as possible} \cite{kitawaki1991sub} & ?? ms & ?? ms & 80 ms \\
\hline
\emph{Take turns verifying random numbers as quickly as possible} \cite{kitawaki1991sub} & ?? ms & ?? ms & 120 ms \\
\hline
\emph{Take turn verifying city names as quickly as possible} \cite{kitawaki1991sub} & ?? ms & ?? ms & 180 ms \\
\hline
\emph{Free conversation} \cite{kitawaki1991sub} & ?? ms & ?? ms & 200 ms \\
\hline
\emph{example} [?] & ?? ms & ?? ms & ?? ms \\

\hline
\textbf{Visual Feedback} & \textbf{150 - 500 ms} & \textbf{500 - 1000 ms} & ?? ms \\
\hline
\emph{example} [?] & ?? ms & ?? ms & ?? ms \\
\hline
\emph{example} [?] & ?? ms & ?? ms & ?? ms \\
\hline
\emph{example} [?] & ?? ms & ?? ms & ?? ms \\
\hline
\emph{example} [?] & ?? ms & ?? ms & ?? ms \\
\hline
\emph{example} [?] & ?? ms & ?? ms & ?? ms \\
\hline
\emph{example} [?] & ?? ms & ?? ms & ?? ms \\
\hline

\end{tabular}
\caption{The three synchronization levels.}
\label{tab:table_synchronization_levels}
\end{table}

[NOTE from LZP] In \cite{schmitt2014asymmetric}, the trial only has three delays, 0ms, 500ms and 1000ms. But it has three kinds of delay, symmetric, moderator and random. Meanwhile, paper[64] is similar to this one.
In \cite{tam2012video}, participants rate the conversation on five-point scale items, but the results are all between 3.5 and 4.5. So I pick the biggest delay as the MOS 3.5 delay.
For the rest of examples in conversation section, I can't summarize them. And in \cite{kitawaki1991sub}, the MOS is in the range of 0 to 4, but I think it is the same as 0 to 5.


\subsection{1. Synchronous Interaction}


\subsubsection{synchronous gesture}

If two distributed users have to gesture exactly at the same time, they may be able to perceive delay like checking their own movement. As \cite{nielsen1993response} explained, 100 ms is an upper boundary for users to fell that the system is running instantaneously. For a better performance, a delay of 30 to 50 ms is needed \cite{chen2007review}.

Imaging that a pair is playing rock-paper-scissors in a 3DTI system with a delay of 100 ms. They expect to perform the gesture exactly at the same time. However, at least one player will find that his partner gesture at least 100 ms slower, which may cause annoyance.

\subsubsection{synchronous speaking}

A pair of users would be sensitive to delay if they have to speak at the same time. For example, when counting down together in a 3DTI system with a delay of 100 ms, at least one user will hear repeated sounds more than 100 ms apart. As a reference, the human ear can distinguish an echo from the original direct sound if the delay is more than 100 ms \cite{wolfel2009distant}. Notice that synchronous speaking is different from a conversation (turn talking).

\subsubsection{instrument ensemble}

With the development of the high-quality network, it is clear that networked music performance has a future \cite{carot2007networked}. 3DTI makes these tasks more nature. For example, two distributed musicians can practice piano duet through 3DTI.

A realistic musical interaction assumes a one-way delay of less than 25 ms \cite{carot2007network}. Beyond this threshold, the groove-building-process cannot be realized by musicians. \cite{schuett2002effects} suggests a delay between 10 - 20 ms for providing a stabilizing effect on the tempo. For a relatively worse network, a coping strategy was discovered that allowed the performers to maintain a solid tempo up to 50 - 70 ms of delay.

\subsection{2. Conversation}

Conversation is an important cue for delay perception. In face-to-face situations, we have learned to unconsciously manage a conversation using the timing of the small pauses in speech \cite{sacks1978simplest}.

[TODO] Theory: Turn Talking Model.

\subsubsection{Chatting}

[TODO]

\subsubsection{Remote guidance}

[TODO]

\subsection{3. Visual Feedback}

[TODO] A short introduction.

[TODO] Theory: Situation Awareness Theory.

[TODO] Theory: Grounding Theory.

\subsubsection{Silent collaboration}

[TODO] Example: Surgery Simulation.

\subsubsection{Imitation}

[TODO] Example: Building Block.

\subsubsection{Turn-based game}

[TODO] Example: Playing chess.

[NOTE] Most actual networks will not exceed such a large delay \cite{donovan2014understanding} [?, ?]

\subsection{[NOTE] Suggestion for network design}

\begin{enumerate}
    \item An application should reach the delay which leads to MOS of 3.5 points.
    \item If already within the noticeable delay, we can appropriately increase the delay to have more room for smoothing and recovering packet loss.
    \item Assistant synchronization can be integrated in an application. For example, we can use synchronized flickers in both side to help a Rock-Paper-Scissors game.
\end{enumerate}

\subsection{[NOTE] Examples of Applications}

\begin{enumerate}
    \item \emph{Rock-paper-scissors}:
    \item \emph{Piano Duet}:
    \item \emph{Chorus}:
    \item \emph{Countdown Together}:
    \item \emph{Chat}:
    \item \emph{Tell-a-lie Game}:
    \item \emph{Building Blocks}:
    \item \emph{Interview}:
    \item \emph{Playing Chess}:
    \item \emph{Building Blocks without Chatting}:
    \item \emph{Magic The Gathering}:
    \item \emph{3D version of Hearthstone}:
    \item \emph{Surgery Simulation}:
    \item \emph{Playing Chess without Seeing Your Partner}:
    \item \emph{Real-time teaching}
    \item \emph{dancing}
\end{enumerate}

%craft written by HBJ
We present an empirical framework of delay perception of 3D Tele Immersion. Tasks in 3DTI can be assigned to 3 different levels: tasks with enforced synchronous interaction, tasks with free conversation and tasks with visual-only communication. Tasks of different levels have different requirements of delay, as shown in figure 1(表格)

There are two indicators of delay measured by most researches: noticeability and disruptiveness of delay. 

Noticeability Noticeability of delay means whether user can perceive the transmission delay in the system or not. In common case if they can communicate very fluently in the visual space as if they were communicating in the real world, then it means they can barely notice the delay. 

Disruptiveness Disruptiveness means whether user can tolerate the delay in the system or not. If they believe the delay hinder them from interacting fluently or the delay disrupt the fairness of certain task or game, the disruptiveness will be very severe.

Countless work has been done on measuring these two factors in previous researches. However, all of them are conducted in audio-only experiments and 2D video experiments, as shown in figure 2(表格介绍以往工作). In audio-only telecommunication research, it is widely acknowledged that the noticeability and tolerance of delay are around 150ms and 400ms (引用). As for audiovisual experiments noticeability and tolerance of delay are around 200ms and 500ms. (need citation)

However, the framework of delay in 3DTI tasks is different from both of them. According to our experiment result, tasks can be divided into three levels. In each level, tasks have similar threshold of delay noticeability due to their common audio and visual factors, while threshold of delay tolerance usually vary in a range because of other factors.

Enforced synchronous interaction The common feature of these tasks is that they all require users to do something at the very same time. Users in this kind of tasks tend to notice delay easily through observing partner’s behavior. Besides, high delay is usually intolerable because it will have obvious negative impact on QoE and fairness with game. Delay noticeability is around 50ms, which is much lower than the limit of 2D video interaction. Delay tolerance vary from 50ms to 150ms.

Some typical examples of enforced synchronous interaction are Rock-paper-scissors game, real-time musical collaboration and count-down game. 
Rock-paper-scissors require both sides of users to show their hand gestures simultaneously. If delay between two sides grows higher than the threshold of noticeability, user will perceive that his or her partner seems to show gestures later than expected. If it exceeds the tolerance delay, then both players will suspect that their partner shows gestures slowly intentionally and the fairness with the game is broken. More details of this game will be described in our experiments.
Real-time musical collaboration requires musicians to work on the same pace. It has an even higher limitation on delay tolerance because if one of the musician lags behind or go ahead just a little fraction of a musical note (say a quaver), the general harmony of music will be destroyed completely. Thus, noticeability and tolerance are almost the same in this task.
Count-down game demands that both players to count from 10 to 1 simultaneously. This is a rather low-requirement task of level 1 because players can tolerate delay up to 150ms. (evidence?)

Visual-only Interaction the common feature of tasks in this level is that audio communication is not available through the whole process. Users can only use body gestures to interact with each other. These tasks are not possible in former 2DTI systems because users cannot express their meaning fully and fluently through a 2D video screen. However, in 3D visual space, since users can see each other in a 3D shape, observe body gestures more easily and understand each other more quickly. Noticeability and tolerance of tasks in this level is much higher than the other 2(3?) levels because users usually work in a slow pace in these tasks and without audio signals they hardly perceive the system delay. Delay noticeability is around 1000ms and tolerance is 2000ms.

Some typical examples are chess game, Reversi game and ?
Chess game involves two players who can play the game without talking to each other. At the beginning of chess games, players can perceive delay more easily because they move and change turns frequently. But as games go on, it takes more and more time for them to think. The slower the game pace is, the less indicators of delay they can perceive.
Reversi game is similar to chess game but it requires more moves. When one side of the player moves, the other side will help remove discarded dots and put on new dots. 



