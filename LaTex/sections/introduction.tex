\section{Introduction}

% [writing] First, provide some context to orient those readers who are less familiar with your topic and to establish the importance of your work.

% [paragraph] development of DIME in audio & 2D
The past centuries have witnessed the growth of communication technology. The invention of the telephone has saved a great deal of time and money by displacing meeting. Recently, Distributed Interactive Multimedia Environments (DIMEs) is getting popular. It provides convenience for teleconference \cite{marlow2016beyond}, tele-collaboration \cite{donovan2014understanding, avellino2015accuracy}, robotic telepresence \cite{jouppi2001robotic, misawa2015chameleonmask, neustaedter2016beam}, and so on. Development of communication technology is never separated from the studies of user experience. For example, delay of 150 ms provides a good user experience for most audio-mediated applications \cite{recommendation2003114, donovan2014understanding}. It has become an industrial standard that contributes to telephone network engineering \cite{itu2003recommendation}. 2D DIME also benefits from the studies of user experience, e.g., in the regions of pointing \cite{higuchi2015immerseboard, avellino2015accuracy}, distance perception \cite{boustila2015evaluation, alexandrova2010egocentric} and delay perception \cite{geerts2011we, tam2012video, schmitt2014influence}.

% [paragraph] rapid growth of 3DTI recently
3DTI emerged in the last past decades \cite{kurillo2008immersive, petit2010multicamera, maimone2011encumbrance, maimone2012real}. Microsoft Research's Holoportation \cite{orts2016holoportation} was impressive. They presented an end-to-end 3DTI pipeline with high-quality, real-time reconstructions of an entire space. Furthermore, parallel computing devices such as GPUs are getting more powerful. Immersive displays such as Head-Mounted Displays (HMDs) are becoming popular. In a word, both the improvements of algorithm and hardware make 3DTI hopeful to be practical in the near future.

% [writing] Second, state the need for your work, as an opposition between what the scientific community currently has and what it wants.

% [paragraph] what the scientific community currently has and what it wants
The focus of previous 3DTI works has been mainly on technical implementations. However, few works were conducted to study user experience in 3DTI. In particular, no work has been done to study users' perception of network delay in an advanced 3DTI system.

% [paragraph] the need of our work
Network delay is a crucial factor that affects user experience \cite{brunnstrom2013qualinet, schmitt2014asymmetric, schmitt2013qoe, wu2009quality}. The studies of delay perception is useful. Numerous works have been carried out to explore delay perception in telephone and 2D DIME. Given a specific task, noticeability and acceptance of delays are important factors to measure \cite{wu2009quality, schmitt2014influence, geerts2011we, schmitt2014asymmetric}. On the one hand, a network service should achieve the acceptable delay as far as possible. On the other hand, there is no need to improve the network while the delay is already unnoticeable. Beyond that, many other strategies were proposed to save network resources and improve the user experience.

% [paragraph] The difference between 3D and 2D - why should we rebuild the framework
Despite of the sufficient research in 2D, it is still necessary to rebuild the framework of delay perception in 3D. The reason is the large difference between 3DTI and 2D DIME: First, 3DTI can support more applications and improve some existing tasks in a more nature manner. We have to discuss them case by case; Second, 3DTI refers to a higher level of immersion, which offers more visual cues. Previous work \cite{tam2012video} have pointed out that video increase users' tolerance to delay. This effect may be enhanced in 3D.

% [writing] Third, indicate what you have done in an effort to address the need (this is the task).

% [paragraph] in this paper, we propse the framework
In this paper, we propose a conceptual framework of network delay perception in 3DTI. It levels delay requirements of 3DTI tasks into three classes: \emph{synchronous} tasks, \emph{audiovisual} tasks and \emph{visual only} tasks, which require network delays of about 50 ms, 200 ms and 400 ms respectively. We designed the framework through a comprehensive review on delay perception and 3DTI systems. For each level in our framework, we summarized suggestions on network engineering. To validate the framework, we first followed the mainstream works to build up our 3DTI systems. Then, we conducted a controlled study to illustrate our framework. To our knowledge, we are the first to investigate users' perception of network delay in a full 3D tele-immersion system.

% [paragraph] Contributions
The contribution of our work is threefold: First, the framework infers a significant change of network delay perception in 3D. We recommend that the 3DTI developers should first assess his application through our framework; Second, we summarize suggestions on network engineering to cope with different level of tasks. These suggestions can help saving network resource and improving the user experience; Third, our project is open-source [?]. We give necessary explanation in the system overview to make sure that the readers can easily build up a similar system.

% [writing] Finally, preview the remainder of the paper to mentally prepare readers for its structure, in the object of the document.

% [paragraph] paper structure
We construct the paper as follow: In section 2, we present our framework. In section 3, we give an overview of our experimental system. In section 4, we describe the controlled study to illustrate our framework. In section 5, we supplement related works on system implementation and existing works on 3DTI delay perception. In section 6, we discuss the limitation of our work. At the end we draw the conclusion.
