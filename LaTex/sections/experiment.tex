\section{User Experiment}

There are two motivations behind our study: first, we use this experiment to explain the four synchronization levels in our framework; second, we illustrate how to measure the noticeable delay and the acceptable delay for a specific application. The experiment is divided into two parts: playing chess and a the Rock-Paper-Scissors game.

\subsection{Part A: Playing Chess}



\subsection{Part B: Rock-Paper-Scissors}

[NOTE] a within study to test the four synchronization levels in our systems. 15 couples, with a total of 30 people take part in. Each participant tries all tasks, we used Latin square to balance the learning effect. For each task, participants had 5 sessions with different end-to-end delay, each of which was followed by short surveys asking participants to rate the experience. The motivation of this study is to investigate noticeability and annoyance of delay in various tasks. 

\subsection{Participants}

[NOTE] A couple of users know each other because they are friends, classmates or something else. User information like age, from campus, payment, whether familiar with AR/VR and telepresence or not.

[NOTE] Our first language is Chinese. This should be claimed because it affects conversion turn talking model. Native Chinese speaker

\subsection{Procedure}

[NOTE] Something the describe the experimental procedure.

After each session we assessed subjective feedback via questionnaires as Table \ref{tab:table_questionnaire} shown. Each session of questions includes perceived quality, noticeability and disruptiveness of delay.

\subsection{Results}

\subsection{NOTE: user experiment}

(PART A) control study for level 2, 3, 4

Duration: 1 hour

3 Tasks (Session) * 5 Delays (Trial)

Tasks = [Audiovisual Playing Chess; Visual Playing Chess; Playing Chess without seeing the partner]

Delays = [100 ms, 200 ms, …, 500 ms]

2 minute for each Trial

1 minute break and questionnaire between Trials

Questions:

(1) Quality: Excellent 1 <—> 5 Bad

(2) Noticeability: Not at all <—> Very much

5 minute break and interview between Sessions

Interview:

(1) How do you notice the delay? What is the cues?

(2) What make you intolerable in the task?

(3) Any commends?

(PART B) control study for level 1

(1) Level 1 (Forced Synchronization Interaction) is with high delay requirement.

(2) Visual assistance is a 0-delay visual cue for a pair to synchronize

Duration: 0.5 hour

2 Tasks * 2 Conditions * 3 Delays

Tasks = [Rock-paper-scissors, counting down]

Conditions = [with, without visual assistance]

Delays = [100 ms, 150 ms, 200 ms]

2 minute for each trial (including questionnaire and rest)

5 minute break and interview between Sessions